\documentclass[11pt]{article}

%%% These are some packages that are useful
\usepackage{amsmath,amssymb, amscd,amsbsy, amsthm, enumerate}
\usepackage[export]{adjustbox}
\usepackage{lastpage}
\usepackage[top=1in, bottom=1in, left=1in, right=1in]{geometry}
\usepackage[unicode]{hyperref}
\usepackage{tikz, pgfplots, xcolor, fancyhdr}
\usepackage{multicol}
\usepackage{lipsum}

%%% Page formatting
%\setlength{\headsep}{30pt}
\setlength{\textheight}{9in}
\newcommand{\tab}{\hspace{1cm}}
%\setlength{\parindent}{25pt}

\title{Is Mathematics Invented or Discovered}
\author{Antonius Torode$^{1,2,3}$, Nat Hawkins$^{1}$, Mike Roosa$^{1,2,3}$, Pranjal Tiwari$^{1,2,3}$ \\ \scriptsize{1. National Superconducting Cyclotron Laboratory, East Lansing, Michigan 48823, USA} \\ \scriptsize{2. Joint Institute for Nuclear Astrophysics, Michigan State University, East Lansing, Michigan 48823, USA} \\ \scriptsize{3. Physics \& Astronomy Department, Michigan State University, East Lansing, Michigan 48823, USA}}

%%% Header and Footer Info
\pagestyle{fancy}
\fancyhead[L]{Potato Pages - VOL I. REF 001.}
\fancyhead[C]{}
\fancyhead[R]{}


\fancyhf{} % sets both header and footer to nothing
\renewcommand{\headrulewidth}{0pt}
% your new footer definitions here

\fancyfoot[L]{}
\fancyfoot[C]{}
\fancyfoot[R]{\thepage\ of \pageref{LastPage}}

% Used to define spacing and format of References
\let\OLDthebibliography\thebibliography
\renewcommand\thebibliography[1]{
	\OLDthebibliography{#1}
	\setlength{\parskip}{0pt}
	\setlength{\itemsep}{0pt plus 0.3ex}
}

%%% Document Starts now
\begin{document}

\maketitle
\thispagestyle{fancy}

\begin{abstract}
	yes.
\end{abstract}

\begin{multicols}{2}
	
\begin{center}
	I. Introduction
\end{center}

Consider a Potato. From this we can determine whether Mathematics is discovered or invented by a simple argument.


\begin{center}
	II. Methods and Results
\end{center}

Consider a conversation amongst colleagues. Denote the colleague A, N, M and P. A conversation emerges and goes as follows: 

\begin{enumerate}
	\item[A] Did y'all know limits were invented 200 years after calculus?!
	\item[P] So what you're saying is, nobody thought about what would happen if something was really really close to 0, but not exactly 0 till like 1900?
	\item[A] More like 1850 but yea...
	\item[N] Shit I didn't know that. That's nuts. But the idea has existed forever. ``Impossible'' is technically a limit. It's the limit of a task as the difficulty goes to infinity.
	\item[A] Well that's the argument of do we invent math or discover math.
	\item[N] Discover Math.	Example: on voyager they used the flipping of spins in hydrogen electrons as the unit of Time because of how well defined it is. The idea being any advanced species would know it. Therefore, they also have to have discovered the same thing we did.
	\item[M] I mean if we're gonna dig into it... Invent math.
	\item[P] I would say discover math from math we invented in the first place.
	\item[M] Our understanding is fundamentally subjective and any scientific measurement or mathematical toll we create is limited by that subjectivity. Although P has a good point about how math progresses.
	\item[N] That's fair. I guess the idea being consistency. Like ants count their steps to navigate. That's been proven. Which means ants also know what ``integers” are. So they found the same thing we did. So did ants invent integers or did we? Or did we both stumble upon the phenomenon of counting? And even ants don't know what an integer is, they know how to count. So we know the same Math for this unit test. Maybe too simple of an example. But I guess I think of it very philosophically. Like evolution. I find it incredibly spiritual to accept our relation to other life. Totally unrelated, but I tend to be more philosophical on these issues.
	\item[M] I think the distinction here comes from our different way of thinking about what kind of THING math is. I think math has to have a practitioner like if there were no people (or ants) there wouldn't be these ideas. From there I think it follows that math can't be a ``universal truth" type thing but more of a ``means of understanding the world" type thing
	The simple examples are probably the best way to go about this type stuff.
	\item[N] Yeah that's fair. But then I suppose that the universe has a nature to it that demands this kind of fundamental tool to understand it. So I'm order to explain nature, we have to find ways of describing it. Thus, discovering Math.
	In that way how do you feel about like (not a potato) but a sunflower that has a spiral in its flower? Did the sun flower discover that structure?
	\item[A] Consider a potato... The potato has a volume. But we invent the notion of dv to express it. Thus math is both discovered and then a formulation of it invented.
	\item[N] That's a good example.	Boom. I like that A.
	\item[P] So the same thing I said.
	\item[M] Also I'm not convinced that "discovery" and "invention" are exclusive (see A's comment).
	\item[N] Yeah pretty much.
	\item[M] Well that was fun. Do we kiss now?
\end{enumerate}

And thus, by following the logic of this argument, mathematics can be argued to be both discovered and invented depending on the sense of the viewer.

\begin{center}
	III. Conclusion
\end{center}

The fundamental principles and ideals shown by mathematics exists without being invented. However, as humans, we must invent notation and formulations to understand and explore mathematics.


\begin{thebibliography}{9}
	{\footnotesize
	\bibitem{ruins} Ancient ruins from the Temple of Baal. 
	}
\end{thebibliography}
\end{multicols}

%%%%%%%%%%%%%%%%%%%%%%%%%%%%%%%%%%%%%%%%%%%%%%%%%%%%%%%%%%%%%%%%%%%%%%%%%%%%%%%%%%%%%%%%%%%
\end{document}





















